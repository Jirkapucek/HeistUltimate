\documentclass[reqno, a4paper]{amsart}
\author{J. Púček, L. Košárková, M. Fuksa}
\usepackage{amsmath}
\usepackage{amssymb}
\usepackage{amsthm}

\usepackage[scale=0.9]{geometry}
\usepackage{mathbbol}

\usepackage[utf8]{inputenc}
\usepackage[czech]{babel}

\usepackage{subfig}
\usepackage{graphicx}
\usepackage{multicol}
\usepackage[font=small,labelfont=bf]{caption}
\usepackage{graphicx,wrapfig}

\input{vit-prusa-macros-experimental}
\newcommand{\navstart}{\mathrm{start}} % subscript denoting start of the navigation (time, initial positions, ...)
\newcommand{\navend}{\mathrm{konec}} % subscript denoting the end of the navigation (time, initial positions, ...)

\newcommand{\navext}{\mathrm{ext}} % extreme value

\title{Pohyb Vzducholod\v{E} v lineárním v\v{E}trném poli}
\renewcommand{\contentsname}{Obsah}
\renewcommand{\abstractname}{Abstrakt}

\begin{document}
\maketitle
\section{Teoretický postup}
\label{sec:teorie}
V případě větrného pole závislého lineárně na pozici se problém hledání nejkratšího letu výrazně zjednoduší. \\
Uvažujme tedy následující větrné pole:
\begin{subequations}
  \label{eq:56}
  \begin{align}
    \label{eq:57}
    u &= - \frac{V}{h}y, \\
    \label{eq:58}
    v &= 0.
  \end{align}
\end{subequations}
Pro toto speciální pole se náš systém diferenciálních rovnic zredukuje na:
\begin{subequations}
  \label{eq:59}
  \begin{align}
    \label{eq:60}
    \dd{x_{\navext}}{t}
    &=
      V \cos \beta_{\navext} - \frac{V}{h}y_{\navext},  \\
    \label{eq:61}
    \dd{y_{\navext}}{t}
    &=
      V \sin \beta_{\navext},  \\
    \label{eq:62}
    \dd{\beta_\navext}{t}
    &=
      \frac{V}{h}
      \cos^2 \beta_\navext
      .
  \end{align}
\end{subequations}
Kde poslední rovnici lze vyřešit explicitně pomocí separace proměnných,
\begin{equation}
  \label{eq:63}
  \tan \beta_{\navext} -  \tan \beta_{\navext, \navend} = \frac{V}{h} (t - t_{\navend, \navext}), 
\end{equation}
kde jsme využili následující značení $\beta_{\navext, \navend} = _{\bydefinition} \left. \beta_{\navext} \right|_{t = t_{\navend, \navext}} $. (Řešení rozepisujeme záměrně tak aby obsahovalo konečný čas, protože ten je co chceme.) Jelikož $\beta_{\navext}$ je ryze rostoucí funkcí času $t$, tak můžeme provést záměnu proměnných a přepsat ~\eqref{eq:61} jako
$
  \dd{y_{\navext}}{\beta_{\navext}} \dd{\beta_{\navext}}{t}
  =
  V \sin \beta_{\navext}
$,
což vede na
\begin{equation}
  \label{eq:64}
  \dd{y_{\navext}}{\beta_{\navext}} = h \frac{\sin \beta_{\navext}}{\cos^2 \beta_{\navext}}.
\end{equation}
Důsledkem toho je, že můžeme také vyřešit rovnici pro $y_{\navext}$
\begin{equation}
  \label{eq:65}
  y_{\navext}(\beta_{\navext})
  =
  h
  \left(
    \frac{1}{\cos \beta_{\navext}}
    -
    \frac{1}{\cos \beta_{\navext, \navend}}
  \right)
  .
\end{equation}
(Pro zjednodušení si určíme konečnou polohu v počátku souřadnic, díky tomu dostáváme $y_{\navext}(\beta_{\navext, \navend}) = 0$.)
Nyní provedeme podobnou záměnu proměnných jako která vedla k rovnici ~\eqref{eq:65}, ale aplikovanou na ~\eqref{eq:60}, dostáváme
\begin{equation}
  \label{eq:66}
  \dd{x_{\navext}}{\beta_{\navext}}
  =
  h
  \left(
    \frac{1}{\cos \beta_{\navext}}
    -
    \frac{1}{\cos^3 \beta_{\navext}}
    +
    \frac{1}{\cos^2 \beta_{\navext} \cos \beta_{\navext, \navend}}
  \right)
  ,
\end{equation}
což implikuje
\begin{equation}
  \label{eq:67}
  x_{\navext}(\beta_{\navext})
  =
-
  \frac{h}{2}
  \left(
    \ln
    \frac{\cos \frac{\beta_{\navext, \navend}}{2} + \sin \frac{\beta_{\navext, \navend}}{2}}{\cos \frac{\beta_{\navext, \navend}}{2} - \sin \frac{\beta_{\navext, \navend}}{2}}
-
    \ln
    \frac{\cos \frac{\beta_{\navext}}{2} + \sin \frac{\beta_{\navext}}{2}}{\cos \frac{\beta_{\navext}}{2} - \sin \frac{\beta_{\navext}}{2}}
    +
    \left(
      \frac{1}{\cos \beta_{\navext}}
      -
      \frac{2}{\cos \beta_{\navext, \navend}}
    \right)
    \tan\beta_{\navext}
+
 \frac{ \tan\beta_{\navext, \navend}}{\cos \beta_{\navext, \navend}}
  \right)
  ,
\end{equation}
v předchozím jsme znovu využily faktu, že konečná poloha je v počátku, takže platí: $x_{\navext}(\beta_{\navext, \navend}) = 0$. Nyní sjednotíme rovnici ~\eqref{eq:67} a ~\eqref{eq:65}. Počáteční pozice musí splňovat $ \left. \vec{x} \right|_{t=t_\navstart} = _{\bydefinition} \vec{x}_{\navstart}$, takže dostáváme následující systém rovnic
\begin{equation}
  \label{eq:68}
  \vec{x}_{\navstart}
  =
  \begin{bmatrix}
 -
  \frac{h}{2}
  \left(
    \ln
    \frac{\cos \frac{\beta_{\navext, \navend}}{2} + \sin \frac{\beta_{\navext, \navend}}{2}}{\cos \frac{\beta_{\navext, \navend}}{2} - \sin \frac{\beta_{\navext, \navend}}{2}}
-
    \ln
    \frac{\cos \frac{\beta_{\navext, \navstart}}{2} + \sin \frac{\beta_{\navext, \navstart}}{2}}{\cos \frac{\beta_{\navext, \navstart}}{2} - \sin \frac{\beta_{\navext, \navstart}}{2}}
    +
    \left(
      \frac{1}{\cos \beta_{\navext, \navstart}}
      -
      \frac{2}{\cos \beta_{\navext, \navend}}
    \right)
    \tan\beta_{\navext, \navstart}
+
 \frac{ \tan\beta_{\navext, \navend}}{\cos \beta_{\navext, \navend}}
\right)
\\
    h
  \left(
    \frac{1}{\cos \beta_{\navext, \navstart}}
    -
    \frac{1}{\cos \beta_{\navext, \navend}}
  \right)
\end{bmatrix}
.
\end{equation}
Jelikož známe $\vec{x}_{\navstart}$, tak je nám jasné, že ~\eqref{eq:68} je systém dvou nelineárních algebraických rovnic pro dvě neznámé: $\beta_{\navext, \navend}$ a $\beta_{\navext, \navstart}$. Jakmile získáme jejich hodnoty, tak už jen stačí použít rovnice ~\eqref{eq:63} k získání konečného času.

Ve výsledku můžeme prohlásit, že k vyřešení problému pro dané $V$, $h$ a $\vec{x}_{\navstart} =_{\bydefinition} \transpose{
  \begin{bmatrix}
    x_{\navstart} &
    y_{\navstart}
  \end{bmatrix}
}$
stačí první vyřešit náš systém nelineárních algebraických rovnic
\begin{subequations}
  \label{eq:69}
  \begin{equation}
    \label{eq:70}
  \begin{bmatrix}
    x_{\navstart} \\
    y_{\navstart}
  \end{bmatrix}
  =
  \begin{bmatrix}
 -
  \frac{h}{2}
  \left(
    \ln
    \frac{\cos \frac{\beta_{\navext, \navend}}{2} + \sin \frac{\beta_{\navext, \navend}}{2}}{\cos \frac{\beta_{\navext, \navend}}{2} - \sin \frac{\beta_{\navext, \navend}}{2}}
-
    \ln
    \frac{\cos \frac{\beta_{\navext}}{2} + \sin \frac{\beta_{\navext}}{2}}{\cos \frac{\beta_{\navext}}{2} - \sin \frac{\beta_{\navext}}{2}}
    +
    \left(
      \frac{1}{\cos \beta_{\navext}}
      -
      \frac{2}{\cos \beta_{\navext, \navend}}
    \right)
    \tan\beta_{\navext}
+
 \frac{ \tan\beta_{\navext, \navend}}{\cos \beta_{\navext, \navend}}
  \right)
\\
    h
  \left(
    \frac{1}{\cos \beta_{\navext, \navstart}}
    -
    \frac{1}{\cos \beta_{\navext, \navend}}
  \right)
\end{bmatrix}
,
\end{equation}
\clearpage
z něho získáme hodnoty $\beta_{\navext, \navstart}$ a $\beta_{\navext, \navend}$. V rovnici
\begin{equation}
  \label{eq:71}
  \tan \beta_{\navext, \navstart} -  \tan \beta_{\navext, \navend} = \frac{V}{h} (t_{\navstart} - t_{\navend, \navext})
\end{equation}
následně najdeme hodnotu konečného času $t_{\navend, \navext}$. Optimální trajektorie je poté jednoznačně určena jako řešení následujícího systému ODR 1. řádu:
  \begin{align}
    \label{eq:72}
    \dd{x_{\navext}}{t}
    &=
      V \cos \beta_{\navext} - \frac{V}{h}y_{\navext},  \\
    \label{eq:73}
    \dd{y_{\navext}}{t}
    &=
      V \sin \beta_{\navext},  \\
    \label{eq:74}
    \dd{\beta_\navext}{t}
    &=
      \frac{V}{h}
      \cos^2 \beta_\navext
      ,
  \end{align}
které řešíme na časovém intervalu $t \in (t_{\navstart}, t_{\navend, \navext})$ v souladu s počátečními podmínkami:
  \begin{align}
    \label{eq:75}
    \left. x_{\navext} \right|_{t= t_{\navstart}} &=  x_{\navstart}, \\
    \label{eq:76}
    \left. y_{\navext} \right|_{t= t_{\navstart}} &=  y_{\navstart}, \\
    \label{eq:77}
    \left. \beta_{\navext} \right|_{t= t_{\navstart}} &=  \beta_{\navext, \navstart}.
  \end{align}
\end{subequations}
Problém~\eqref{eq:69} lze vyřešit standardními numerickými metodami.

\section{Konkrétní řešení v programu Mathematica}
\label{sec:Mathematica}

Pro náš speciální případ pole~\eqref{eq:56} postupujeme, jak je popsáno výše tedy:
\subsection*{Krok 1}
Vyřešíme nelineární algebraickou soustavu dvou rovnic o dvou neznámých $\beta_{\navext, \navstart}$ a $\beta_{\navext, \navend}$. Rovnici ~\eqref{eq:70} vyřešíme pomocí příkazu
\begin{verbatim*}
	FindRoot[{-(h/2)*(Log[(Cos[be/2]+Sin[be/2])/(Cos[be/2]-Sin[be/2])]-
		Log[(Cos[bs/2]+Sin[bs/2])/(Cos[bs/2]-Sin[bs/2])]+(1/Cos[bs]-2/Cos[be])*Tan[bs]+Tan[be]/Cos[be])==x0,
		h*((1/(Cos[bs])-1/Cos[be]))==y0},{{bs,0},{be,0}}],
\end{verbatim*}
kde $be$ a $bs$ jsou značení pro $\beta_{\navext, \navend}$ a $\beta_{\navext, \navstart}$.
\subsection*{Krok 2}
Vypočítáme $t_{\navend, \navext}$ z rovnice ~\eqref{eq:71} pomocí příkázu
\begin{verbatim*}
	Solve[{Tan[bs]-Tan[be]==(V/h)*(ts-te)}/.bsbe,te].
\end{verbatim*}
\subsection*{Krok 3}
Nyní máme všechny počáteční podmínky a stačí vyřešit soustavu rovnic ~\eqref{eq:72} až ~\eqref{eq:74} s počátečními podmínkami ~\eqref{eq:75} až ~\eqref{eq:77}. Využijme příkazu
\begin{verbatim*}
	eqs={x'[t]==V*Cos[b[t]]-(V/h)*y[t],y'[t]==V*Sin[b[t]],b'[t]==(V/h)*(Cos[b[t]])^2}
	
	time={t,ts,Re[te/.tend]}
	
	sol=NDSolve[Join[eqs,{x[0]==x0,y[0]==y0,b[0]==Re[bs/.bsbe]}],{x[t],y[t],b[t]},time].
\end{verbatim*}
\subsection*{Krok 4}
Vykreslíme graf a blaženě pozorujeme, jak se nám to povedlo (viz obrázky).
\begin{verbatim*}
	trajektorie=ParametricPlot[{x[t],y[t]}/.sol,{t,ts,te}, Frame->True,GridLines->Automatic,PlotRange->
	{{-10,10},{-10,10}}]
	
	Show[field,trajektorie]
\end{verbatim*}
\begin{minipage}{\textwidth}
	\begin{minipage}[b]{0.5\textwidth}
		\centering
		\includegraphics[width=0.9\textwidth]{lodka1}
		\captionof*{figure}{\textbf{Obrázek 1.} $x_0=8, y_0=0$}
	\end{minipage}
	\begin{minipage}[b]{0.5\textwidth}
		\centering
		\includegraphics[width=0.9\textwidth]{lodka2}
		\captionof*{figure}{\textbf{Obrázek 2.} $x_0=-2, y_0=5$}
	\end{minipage}
	\hfill
\end{minipage}
\subsection*{Počáteční podmínka $y_0=0$}
Všimněme si, že případ na obrázku (2) přejde pro $y=0$ na úlohu případu obrázku (1). Tento závěr není nijak překvapivý, protože pro tento model nepředpokládáme žádné zpoždění zatáčení a navíc její rychlost je čistě dána její polohou a parametry $V$ a $h$ (tedy v tomto systému nepředpokládáme žádné "nabrání" kinetické energie z předchozí části trajektorie). Rovněž není překvapivé, že pro $y_0=0$ je ideální trajektorie symetrická podle osy $x=\frac{x_0}{2}$. Toto tvrzení si nyní dokažme.\\
\\
(pozn. tento důkaz stojí na předpokladu, že se objekt pohybuje symetricky pro osu $x$, ovšem předpis $x$-ové souřadnice je i v tomto jednoduchém silovém poly velmi složitý, proto ho nebudeme zkoumat a pouze tuto krásnou vlastnost předpokládejme)\\
\\
Z informace $y_0=0$ a vztahu ~\eqref{eq:70} dostáváme
\begin{equation}
	\label{eq:75}
	\frac{1}{\cos\beta_{\navext, \navend}} = \frac{1}{\cos\beta_{\navext, \navstart}} \equiv  \cos\beta_{\navext, \navend} = \cos\beta_{\navext, \navstart}.
\end{equation}
Nyní využijme fyzikálního nadhledu. Jelikož se musíme dostat z počátečního bodu na ose $y=0$ do počátku, jistě musí platit
\begin{equation}
	\label{eq:76}
	\beta_{\navext, \navend}=-\beta_{\navext, \navstart}.
\end{equation}
Dále využijme toho, že $\tan x$ je lichá funkce. S tímto poznatkem a vztahy ~\eqref{eq:71}, ~\eqref{eq:76} dostáváme, že pro $t_{\navend, \navext}$ volbou $t_{\navstart}=0$ platí
\begin{equation*}
	\tan \beta_{\navext, \navstart} -  \tan \beta_{\navext, \navend} = -\tan \beta_{\navext, \navend} -  \tan \beta_{\navext, \navend}=-2\tan \beta_{\navext, \navend}
	=\frac{V}{h} (t_{\navstart} - t_{\navend, \navext})=-\frac{V}{h} t_{\navend, \navext} \implies
\end{equation*}
\begin{equation}
	\label{eq:77}
	t_{\navend, \navext}=\frac{2h}{V}\tan \beta_{\navext, \navend}.
\end{equation}
Opět využijme vztahu ~\eqref{eq:71} a vyjádřeme $\beta$ jako $\beta (t)$
\begin{equation*}
	\tan \beta_\navext = \frac{V}{h}(t-t_{\navend, \navext})+\tan \beta_{\navext, \navend}=\frac{V}{h}(t-\frac{2h}{V}\tan \beta_{\navext, \navend})+\tan \beta_{\navext, \navend}=\frac{V}{h}t - \tan \beta_{\navext, \navend} \implies
\end{equation*}
\begin{equation}
	\label{eq:78}
	\beta_\navext=\arctan(\frac{V}{h}t - \tan \beta_{\navext, \navend}).
\end{equation}
V závěrečné fázi našeho dokazování si pomocí funkce $y(\beta_\navext)$, kterou si vyjádříme jako $y(t)$, ukážeme, že je symetrická podle času $\frac{t_{\navend, \navext}}{2}$, tedy že platí $y(\frac{t_{\navend, \navext}}{2}+\epsilon)=y(\frac{t_{\navend, \navext}}{2}-\epsilon)$.\\
Začněme vyjádřením $y=y(t)$ ze vztahu ~\eqref{eq:65}
\begin{equation}
	\label{eq:79}
	y(\beta_\navext(t))=h
	\left(
	\frac{1}{\cos \beta_{\navext}(t)}
	-
	\frac{1}{\cos \beta_{\navext, \navend}}
	\right)=h
	\left(
	\frac{1}{\cos \arctan(\frac{V}{h}t - \tan \beta_{\navext, \navend})}
	-
	\frac{1}{\cos \beta_{\navext, \navend}}
	\right)=
\end{equation}
\begin{equation*}
	=h
	\left(
	\frac{1}{\cos \arctan(\frac{V}{h}t - \tan \beta_{\navext, \navend})}
	-
	\frac{1}{\cos \beta_{\navext, \navend}}
	\right)=h\left(
	\sqrt{\left( \frac{V}{h}t-\tan \beta_{\navext, \navend}\right)^2 + 1 }
	-
	\frac{1}{\cos \beta_{\navext, \navend}}
	\right) \implies
\end{equation*}
\begin{equation}
	\label{eq:80}
	y(t)=h\left(
	\sqrt{\left( \frac{V}{h}t-\tan \beta_{\navext, \navend}\right)^2 + 1 }
	-
	\frac{1}{\cos \beta_{\navext, \navend}}
	\right).
\end{equation}
Ve výpočtu se nám objevila zajímavá identita $\cos \arctan$, pro kterou platí $\cos \arctan (x)=\frac{1}{\sqrt{x^2+1}}$. Nyní nám už zbývá dosadit hodnotu $t = \frac{ t_{\navend, \navext}}{2}\pm\epsilon$ do $y(t)$.
\begin{equation*}
	y\left( \frac{ t_{\navend, \navext}}{2}\pm\epsilon\right) =h\left(
	\sqrt{\left( \frac{V}{h}\left( \frac{ t_{\navend, \navext}}{2}\pm\epsilon\right) -\tan \beta_{\navext, \navend}\right)^2 + 1 }
	-
	\frac{1}{\cos \beta_{\navext, \navend}}
	\right)=
\end{equation*}
\begin{equation*}
	=h\left(
	\sqrt{\left( \frac{V}{h}\left( \frac{\frac{2h}{V}\tan\beta_{\navext, \navend}}{2}\pm\epsilon\right) -\tan \beta_{\navext, \navend}\right)^2 + 1 }
	-
	\frac{1}{\cos \beta_{\navext, \navend}}
	\right)=
\end{equation*}
\begin{equation}
	\label{eq:80}
	=h\left(
	\sqrt{\left( \tan \beta_{\navext, \navend}\pm\epsilon -\tan \beta_{\navext, \navend}\right)^2 + 1 }
	-
	\frac{1}{\cos \beta_{\navext, \navend}}
	\right)=h\left(\sqrt{\epsilon^2+1}-\frac{1}{\cos \beta_{\navext, \navend}}
	\right)
\end{equation}
Tedy jsme dokázali, že řešení je symetrické podle bodu $x=\frac{x_0}{2}$ respektive podle času $t = \frac{ t_{\navend, \navext}}{2}$, kdy (podle fyzikálních představ) osu $x=\frac{x_0}{2}$ protíná. $\qed$

\end{document}

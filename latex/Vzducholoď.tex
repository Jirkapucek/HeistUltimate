\documentclass[reqno, a4paper]{amsart}
\author{J. Púček, L. Košárková, M. Fuksa}
\usepackage{amsmath}
\usepackage{amssymb}
\usepackage{amsthm}

\usepackage[scale=0.9]{geometry}
\usepackage{mathbbol}

\usepackage[utf8]{inputenc}
\usepackage[czech]{babel}

\usepackage{subfig}
\usepackage{graphicx}
\usepackage{multicol}
\usepackage[font=small,labelfont=bf]{caption}
\usepackage{graphicx,wrapfig}

\input{vit-prusa-macros-experimental}
\newcommand{\navstart}{\mathrm{start}} % subscript denoting start of the navigation (time, initial positions, ...)
\newcommand{\navend}{\mathrm{konec}} % subscript denoting the end of the navigation (time, initial positions, ...)

\newcommand{\navext}{\mathrm{ext}} % extreme value

\title{Pohyb Vzducholod\v{E} v lineárním v\v{E}trném poli}
\renewcommand{\contentsname}{Obsah}
\renewcommand{\abstractname}{Abstrakt}

\begin{document}
\maketitle
\section{Teoretický postup}
\label{sec:teorie}
V případě větrného pole závislého lineárně na pozici se problém hledání nejkratšího letu výrazně zjednoduší. \\
Uvažujme tedy následující větrné pole:
\begin{subequations}
  \label{eq:56}
  \begin{align}
    \label{eq:57}
    u &= - \frac{V}{h}y, \\
    \label{eq:58}
    v &= 0.
  \end{align}
\end{subequations}
Pro toto speciální pole se náš systém diferenciálních rovnic zredukuje na:
\begin{subequations}
  \label{eq:59}
  \begin{align}
    \label{eq:60}
    \dd{x_{\navext}}{t}
    &=
      V \cos \beta_{\navext} - \frac{V}{h}y_{\navext},  \\
    \label{eq:61}
    \dd{y_{\navext}}{t}
    &=
      V \sin \beta_{\navext},  \\
    \label{eq:62}
    \dd{\beta_\navext}{t}
    &=
      \frac{V}{h}
      \cos^2 \beta_\navext
      .
  \end{align}
\end{subequations}
Kde poslední rovnici lze vyřešit explicitně pomocí separace proměnných,
\begin{equation}
  \label{eq:63}
  \tan \beta_{\navext} -  \tan \beta_{\navext, \navend} = \frac{V}{h} (t - t_{\navend, \navext}), 
\end{equation}
kde jsme využili následující značení $\beta_{\navext, \navend} = _{\bydefinition} \left. \beta_{\navext} \right|_{t = t_{\navend, \navext}} $. (Řešení rozepisujeme záměrně tak aby obsahovalo konečný čas, protože ten je co chceme.) Jelikož $\beta_{\navext}$ je ryze rostoucí funkcí času $t$, tak můžeme provést záměnu proměnných a přepsat ~\eqref{eq:61} jako
$
  \dd{y_{\navext}}{\beta_{\navext}} \dd{\beta_{\navext}}{t}
  =
  V \sin \beta_{\navext}
$,
což vede na
\begin{equation}
  \label{eq:64}
  \dd{y_{\navext}}{\beta_{\navext}} = h \frac{\sin \beta_{\navext}}{\cos^2 \beta_{\navext}}.
\end{equation}
Důsledkem toho je, že můžeme také vyřešit rovnici pro $y_{\navext}$
\begin{equation}
  \label{eq:65}
  y_{\navext}(\beta_{\navext})
  =
  h
  \left(
    \frac{1}{\cos \beta_{\navext}}
    -
    \frac{1}{\cos \beta_{\navext, \navend}}
  \right)
  .
\end{equation}
(Pro zjednodušení si určíme konečnou polohu v počátku souřadnic, díky tomu dostáváme $y_{\navext}(\beta_{\navext, \navend}) = 0$.)
Nyní provedeme podobnou záměnu proměnných jako která vedla k rovnici ~\eqref{eq:65}, ale aplikovanou na ~\eqref{eq:60}, dostáváme
\begin{equation}
  \label{eq:66}
  \dd{x_{\navext}}{\beta_{\navext}}
  =
  h
  \left(
    \frac{1}{\cos \beta_{\navext}}
    -
    \frac{1}{\cos^3 \beta_{\navext}}
    +
    \frac{1}{\cos^2 \beta_{\navext} \cos \beta_{\navext, \navend}}
  \right)
  ,
\end{equation}
což implikuje
\begin{equation}
  \label{eq:67}
  x_{\navext}(\beta_{\navext})
  =
  h
  \left(
    \frac{1}{2}
    \ln
    \frac{\cos \frac{\beta_{\navext}}{2} + \sin \frac{\beta_{\navext}}{2}}{\cos \frac{\beta_{\navext}}{2} - \sin \frac{\beta_{\navext}}{2}}
    -
    \frac{1}{2}
    \ln
    \frac{\cos \frac{\beta_{\navext, \navend}}{2} + \sin \frac{\beta_{\navext, \navend}}{2}}{\cos \frac{\beta_{\navext, \navend}}{2} - \sin \frac{\beta_{\navext, \navend}}{2}}
    +
    \left(
      \frac{1}{\cos \beta_{\navext, \navend}}
      -
      \frac{1}{\cos \beta_{\navext}}
    \right)
    \frac{\sin \beta_{\navext}}{\cos \beta_{\navext}}
  \right)
  ,
\end{equation}
v předchozím jsme znovu využily faktu, že konečná poloha je v počátku, takže platí: $x_{\navext}(\beta_{\navext, \navend}) = 0$. Nyní sjednotíme rovnici ~\eqref{eq:67} a ~\eqref{eq:65}. Počáteční pozice musí splňovat $ \left. \vec{x} \right|_{t=t_\navstart} = _{\bydefinition} \vec{x}_{\navstart}$, takže dostáváme následující systém rovnic
\begin{equation}
  \label{eq:68}
  \vec{x}_{\navstart}
  =
  \begin{bmatrix}
  h
  \left(
    \frac{1}{2}
    \ln
    \frac{\cos \frac{\beta_{\navext, \navstart}}{2} + \sin \frac{\beta_{\navext, \navstart}}{2}}{\cos \frac{\beta_{\navext, \navstart}}{2} - \sin \frac{\beta_{\navext, \navstart}}{2}}
    -
    \frac{1}{2}
    \ln
    \frac{\cos \frac{\beta_{\navext, \navend}}{2} + \sin \frac{\beta_{\navext, \navend}}{2}}{\cos \frac{\beta_{\navext, \navend}}{2} - \sin \frac{\beta_{\navext, \navend}}{2}}
    +
    \left(
      \frac{1}{\cos \beta_{\navext, \navend}}
      -
      \frac{1}{\cos \beta_{\navext, \navstart}}
    \right)
    \frac{\sin \beta_{\navext, \navstart}}{\cos \beta_{\navext, \navstart}}
  \right)
\\
    h
  \left(
    \frac{1}{\cos \beta_{\navext, \navstart}}
    -
    \frac{1}{\cos \beta_{\navext, \navend}}
  \right)
\end{bmatrix}
.
\end{equation}
Jelikož známe $\vec{x}_{\navstart}$, tak je nám jasné, že ~\eqref{eq:68} je systém dvou nelineárních algebraických rovnic pro dvě neznámé: $\beta_{\navext, \navend}$ a $\beta_{\navext, \navstart}$. Jakmile získáme jejich hodnoty, tak už jen stačí použít rovnice ~\eqref{eq:63} k získání konečného času.

Ve výsledku můžeme prohlásit, že k vyřešení problému pro dané $V$, $h$ a $\vec{x}_{\navstart} =_{\bydefinition} \transpose{
  \begin{bmatrix}
    x_{\navstart} &
    y_{\navstart}
  \end{bmatrix}
}$
stačí první vyřešit náš systém nelineárních algebraických rovnic
\begin{subequations}
  \label{eq:69}
  \begin{equation}
    \label{eq:70}
  \begin{bmatrix}
    x_{\navstart} \\
    y_{\navstart}
  \end{bmatrix}
  =
  \begin{bmatrix}
  h
  \left(
    \frac{1}{2}
    \ln
    \frac{\cos \frac{\beta_{\navext, \navstart}}{2} + \sin \frac{\beta_{\navext, \navstart}}{2}}{\cos \frac{\beta_{\navext, \navstart}}{2} - \sin \frac{\beta_{\navext, \navstart}}{2}}
    -
    \frac{1}{2}
    \ln
    \frac{\cos \frac{\beta_{\navext, \navend}}{2} + \sin \frac{\beta_{\navext, \navend}}{2}}{\cos \frac{\beta_{\navext, \navend}}{2} - \sin \frac{\beta_{\navext, \navend}}{2}}
    +
    \left(
      \frac{1}{\cos \beta_{\navext, \navend}}
      -
      \frac{1}{\cos \beta_{\navext, \navstart}}
    \right)
    \frac{\sin \beta_{\navext, \navstart}}{\cos \beta_{\navext, \navstart}}
  \right)
\\
    h
  \left(
    \frac{1}{\cos \beta_{\navext, \navstart}}
    -
    \frac{1}{\cos \beta_{\navext, \navend}}
  \right)
\end{bmatrix}
,
\end{equation}
\clearpage
z něho získáme hodnoty $\beta_{\navext, \navstart}$ a $\beta_{\navext, \navend}$. V rovnici
\begin{equation}
  \label{eq:71}
  \tan \beta_{\navext, \navstart} -  \tan \beta_{\navext, \navend} = \frac{V}{h} (t_{\navstart} - t_{\navend, \navext})
\end{equation}
následně najdeme hodnotu konečného času $t_{\navend, \navext}$. Optimální trajektorie je poté jednoznačně určena jako řešení následujícího systému ODR 1. řádu:
  \begin{align}
    \label{eq:72}
    \dd{x_{\navext}}{t}
    &=
      V \cos \beta_{\navext} - \frac{V}{h}y_{\navext},  \\
    \label{eq:73}
    \dd{y_{\navext}}{t}
    &=
      V \sin \beta_{\navext},  \\
    \label{eq:74}
    \dd{\beta_\navext}{t}
    &=
      \frac{V}{h}
      \cos^2 \beta_\navext
      ,
  \end{align}
které řešíme na časovém intervalu $t \in (t_{\navstart}, t_{\navend, \navext})$ v souladu s počátečními podmínkami:
  \begin{align}
    \label{eq:75}
    \left. x_{\navext} \right|_{t= t_{\navstart}} &=  x_{\navstart}, \\
    \label{eq:76}
    \left. y_{\navext} \right|_{t= t_{\navstart}} &=  y_{\navstart}, \\
    \label{eq:77}
    \left. \beta_{\navext} \right|_{t= t_{\navstart}} &=  \beta_{\navext, \navstart}.
  \end{align}
\end{subequations}
Problém~\eqref{eq:69} lze vyřešit standardními numerickými metodami.

\section{Konkrétní řešení v programu Mathematica}
\label{sec:Mathematica}


\end{document}
